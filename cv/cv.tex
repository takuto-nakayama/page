\documentclass[a4paper,11pt]{article}

\usepackage[margin=2.5cm]{geometry}
\usepackage{titlesec}
\usepackage{enumitem}
\usepackage{hyperref}
\usepackage{luatexja-fontspec}
\usepackage{fontspec}
\usepackage{enumitem}
\usepackage{makecell}

% フォント
\setmainfont{Times New Roman}
\setsansfont{Arial}
\setmonofont{Courier New}
\newfontfamily\ipamincho[
  Path = /Users/takuto/Library/Fonts/,
  UprightFont = ipaexm.ttf,
]{IPAexMincho}
% セクションタイトル設定
\titleformat{\section}{\large\bfseries}{}{0em}{}[\titlerule]

% ページ番号なし
\pagestyle{empty}

%------------------------------%

\begin{document}

\begin{center}
  {\LARGE \textbf{Takuto NAKAYAMA}}\\
  \vspace{0.5em}
  \href{mailto:takuto@example.com}{tnakayama.a5ling@gmail.com} \quad | \quad +81-90-1846-8451 \\
  \vspace{0.2em} 
\end{center}

\vspace{1em}

\section*{Education}
\renewcommand{\arraystretch}{1.2}
\begin{tabular}{@{}ll}
  \textbullet\hspace{0.5em} B.A. in Literature & 2020 March at Keio University, Tokyo, Japan \\
  \textbullet\hspace{0.5em} M.A. in Literature & 2022 March at Keio University, Tokyo, Japan \\
  \textbullet\hspace{0.5em} Ph.D. candidate & 2025 February at Keio University, Tokyo, Japan \\
  \textbullet\hspace{0.5em} Ph.D. in Literature & expected 2026 March at Keio University, Tokyo, Japan \\
  \end{tabular}
  \renewcommand{\arraystretch}{1}
  
\vspace{1em}

\section*{Research Interests}
\begin{itemize}[leftmargin=*, itemsep=0em, topsep=0em]
  \item Computational Linguistics
  \item Quantitative Linguistics
  \item Mathematical Linguistics
  \item Linguistic Complexity
\end{itemize}

\vspace{1em}

\section*{Languages}
  \hspace{0.5em} Japanese (native), English (fluent), German (beginner)

\vspace{1em}

\section*{Skills}
  \hspace{0.5em} Python, \LaTeX

\vspace{1em}

\section*{Publications}
\subsection*{Research Articles}
\begin{itemize}[leftmargin=*, itemsep=0em, topsep=0em]
  \item Nakayama, T. (in press.). 言語の複雑さを「測る」ことの正体-測度論基盤の目論みと測る対象としての言語 [What ``measuring'' linguistic complexity is: Measure theoretical foundation and a language as a measurable object]. \textit{Fora}, \textit{8}. 
  \item  Nakayama, T. (2024). How Skewed is the World for a Language?: A Computational Approach to the Semantic Distribution of Languages, \textit{Kyorin University Journal}, \textit{41}, 61–72. 
  \item Nakayama, T. (2023). Is a language diachronically equally complex?: An information theory approach to linguistic complexity, \textit{Kyorin University Journal}, \textit{40}, 75–87. 
  \item Nakayama, T. (2021). How do speakers communicate with different linguistic knowledge?: The internal and external stabilizers of a language, \textit{Colloqia}, \textit{42}, 51–62.
  \item Nakayama, T. (2020). Causality and resultantity in language: Language in individuals and groups, \textit{Colloquia}, \textit{41}, 83–98.
\end{itemize}

\subsection*{Proceedings}
\begin{itemize}[leftmargin=*, itemsep=0em, topsep=0em]
  \item 
\end{itemize}

\vspace{1em}

\section*{Conference Talks}
\begin{itemize}[leftmargin=*, itemsep=0em, topsep=0em]
  \item Nakayama, T. (2024, March 11). 言語一般の計測を目指して: サブワードと分散意味論に基づく言語の複雑性計測 [\textit{Toward a linguiscitally general measurement: Measuring linguistic complexity based on subword tokens and distributional semantics}]. [poster] The 31st Annual Conference of the Association for Natural Language Processing, Nagasaki, Japan.
  \item Nakayama, T. (2024, February 28). 言語は等しく多義的か?–サブワードと分散意味論に基づく形式–意味対応の分析 [\textit{Are languages equally polysemous?: An analysis of form-meaning pairing based on subword and distributional semantics}]. [oral presentation] The 49th Annual Conference of the Japanese Association of Sociolinguistic Sciences, Tokyo, Japan.
  \item Nakayama, T. (2024 September, 10). \textit{Linguistic complexity through form-meaning pairings: An information theoretical approach to equi-complexity of language} [oral presentation].The 21st International Congress of Linguists, Poznań, Poland.
  \item Nakayama, T. (2024, March). 言語は等しく複雑か?: 多義語埋め込み表現による形式-意味対応の複雑性 [\textit{Is languages equally complex?: Complexity of form-meaning pairings through polysemous word enbeddings}]. [poster] The 30th Annual Conference of the Association for Natural Language Processing, Kobe, Japan.
  \item Nakayama, T. (2023 August 8). \textit{The equi-complexity vs. typology: Measurement of overall linguistic complexity and typological categories} [poster]. International Cognitive Linguistics Conference 16, Düsseldorf, Germany.
  \item Nakayama, T. (2023 June 29). \textit{Are all languages equally complex?: Information theory-based method to measure the overall complexity of a language} [poster]. Quantitative Linguistics Conference 2023, Lausanne, Switzerland.
  \item Nakayama, T. (2022 July 16). \textit{Convergence and divergence in language: How do mutual understandings among speakers emerge from iterated interactions?} [panel discussion]. Sociolinguistics Symposium 24, Ghent, Belgium.
  \item Nakayama, T. (2021, June 28). \textit{A convergence of language between child and mother: Semantic Analysis based on FCA} [panel discussion]. 17th International Pragmatics Association, Winterthur, Switzerland.
\end{itemize}

\vspace{1em}

\section*{Work Experience}
\renewcommand{\arraystretch}{1.5}
\begin{tabular}{@{}lll}
  \textbullet\hspace{0.5em} \makecell[lt]{April 2024 – \\ \; March 2026} & research fellow & Japan Society for the Promotion of Science \\
  \textbullet\hspace{0.5em} April 2024 –  & English (part-time lecturer) & Kyoritsu Women's University, Tokyo \\
  \textbullet\hspace{0.5em} April 2024 –  & TOEFL English (part-time lecturer) & Kyoritsu Women's University, Tokyo \\
  \textbullet\hspace{0.5em} \makecell[lt]{April 2023 – \\ \; March 2024} & editor-in-chief & \textit{Colloquia} \\
  \textbullet\hspace{0.5em} April 2022 –  & English (part-time lecturer) & Kyorin University, Tokyo \\
  \textbullet\hspace{0.5em} April 2022 –  & Medical English (part-time lecturer) & Kyorin University, Tokyo
\end{tabular}
\renewcommand{\arraystretch}{1}

\vspace{1em}

\section*{Grants}
\renewcommand{\arraystretch}{1.2}
\begin{tabular}{@{}ll}
  \textbullet\hspace{0.5em} 2024–2026 & Research Fellow (DC2), Japan Society for the Promotion of Science \\
  \textbullet\hspace{0.5em} 2022–2024 & Support for Pioneering Research Initiated by the Next Generation, Japan Science and Technology Agency \\
\end{tabular}
\renewcommand{\arraystretch}{1}

\vspace{1em}

\section*{Memberships}
\renewcommand{\arraystretch}{1.2}
\begin{tabular}{@{}ll}
  \textbullet\hspace{0.5em} 2024 – & The Association for Natural Language Processing   \\
  \textbullet\hspace{0.5em} 2023 – & International Quantitative Linguistics Association \\
  \textbullet\hspace{0.5em} 2022 – & Japan Association of Sociolinguistic Science \\
\end{tabular}
\renewcommand{\arraystretch}{1}

\end{document}
