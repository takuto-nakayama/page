\documentclass[11pt,a4paper,sans]{moderncv}

% スタイルとカラー
\moderncvstyle{banking}      % 'casual', 'classic', 'oldstyle', 'banking' など
\moderncvcolor{blue}         % 'blue', 'orange', 'green', 'red', 'purple', 'grey', 'black'

% フォント設定(日本語を使うなら XeLaTeX か LuaLaTeX + fontspec が必要)
\usepackage[scale=0.8]{geometry}
\usepackage{fontspec}
\setmainfont{Times New Roman}
%\setmainjfont{IPAexMincho}   % 日本語フォント(例)

\usepackage{xcolor}             % カラーを定義するため
\definecolor{mycolor}{HTML}{042454}

% 個人情報
\name{Takuto}{Nakayama}
\title{Curriculum Vitae}
\address{Tokyo, Japan}{}{}
\phone[mobile]{+81~90~1846~8451}
\email{tnakayama.a5ling@gmail.com}
\photo[64pt][0.4pt]{selfie_circled.png} % 写真(オプション)

\begin{document}

\makecvtitle

\section{Education}
	\cventry
		{2020--2022}
		{M.A. in Literature}
		{Keio University}
		{Tokyo, Japan}{}{}
	\cventry
		{2016--2020}
		{B.A. in Literature}
		{Keio University}
		{Tokyo, Japan}{}{}

\section{Languages}
	\cvitemwithcomment{Japanese}{Native}{}
	\cvitemwithcomment{English}{Fluent}{}
	\cvitemwithcomment{German}{Basic}{Studied for 2 years}

\section{Skills}
	\cvitem{Programming}{Python}

\section{Publications}
	\cvitem{2024}{\textit{“Exploring Japanese WordPiece Tokenization in BERT Models”}, ICL, Poland}

\section{Presentation}

\section{}
\end{document}
